%% Chinese and English title %%
% 封面论文题目. 换行使用换行符("\\")
\title{
    中山大学 \\
    本科毕业论文非正式模版
}
\ctitle{中山大学本科毕业论文非正式模版}     % 的图像语义分割???
\etitle{Unofficial LaTex Template for Undergraduate Thesis of SYSU}% 英文论文题目
%% Detail information of the author
\author{陈冠英}%作者
\cauthor{陈\ 冠\ 英}%封面上作者(各字符间可加空白)
\eauthor{Chen Guanying}%封面上作者(各字符间可加空白)
\date{二〇一六年四月}%完成日期
\studentid{12350004}%学号
\cschool{电子与信息工程学院}%院(系)
\cmajor{自动化}%专业
\emajor{Automation}
\cmentor{林倞\ 教授}%指导老师
\ementor{Prof. Lin Liang}%指导老师

%% Chinese and English keywords %%
\ckeywords{本科毕业论文;LaTex模板;中山大学}%中文关键词(每个关键词之间用“;”分开,最后一个关键词不打标点符号。)
\ekeywords{undergraduate thesis, LaTex template, Sun Yat-Sen University}%英文关键词

%% all kinds of tables %%

%% Chinese and English abstract %%
%% 1. state the problem, your approach and solution, and the main contributions of the paper. Include little if any background and motivation. Be factual but comprehensive. The material in the abstract should not be repeated later word for word in the paper.
\cabstract{
%state the problem
摘要内容应概括地反映出本论文的主要内容,主要说明本论文的研究目的、内容、方法、成果和结论。要突出本论文的创造性成果或新见解,不要与引言相混淆。语言力求精练、准确,以300—500字为宜。在摘要的下方另起一行,注明本文的关键词(3—5个)。关键词是供检索用的主题词条,应采用能覆盖论文主要内容的通用技术词条(参照相应的技术术语标准)。按词条的外延层次排列(外延大的排在前面)。摘要与关键词应在同一页。
}
\eabstract{
英文摘要内容与中文摘要相同,以250—400个实词为宜。摘要下方另起一行注明英文关键词(Keywords3—5个)。
}
\endinput
