%%
% 致谢
% 谢辞应以简短的文字对课题研究与论文撰写过程中曾直接给予帮助的人员(例如指导教师、答疑教师及其他人员)表示对自己的谢意,这不仅是一种礼貌,也是对他人劳动的尊重,是治学者应当遵循的学术规范。内容限一页。
% modifier: 黄俊杰
% update date: 2017-04-15
%%

\chapter{致谢}

	四年时间转眼即逝,青涩而美好的本科生活快告一段落了。回首这段时间,我不仅学习到了很多知识和技能,而且提高了分析和解决问题的能力与养成了一定的科学素养。虽然走过了一些弯路,但更加坚定我后来选择学术研究的道路,实在是获益良多。这一切与老师的教诲和同学们的帮助是分不开的,在此对他们表达诚挚的谢意。

	首先要感谢的是我的指导老师王大明教授。我作为一名本科生,缺少学术研究经验,不能很好地弄清所研究问题的重点、难点和热点,也很难分析自己的工作所能够达到的层次。王老师对整个研究领域有很好的理解,以其渊博的知识和敏锐的洞察力给了我非常有帮助的方向性指导。他严谨的治学态度与辛勤的工作方式也是我学习的榜样,在此向王老师致以崇高的敬意和衷心的感谢。

	最后我要感谢我的家人,正是他们的无私的奉献和支持,我才有了不断拼搏的信息的勇气,才能取得现在的成果。

\vskip 108pt
\begin{flushright}
	王小明\makebox[1cm]{} \\
\today
\end{flushright}

