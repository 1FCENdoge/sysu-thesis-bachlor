%%
% 开题报告
% modifier: 黄俊杰(huangjj27, 349373001dc@gmail.com)
% update date: 2017-05-14

% 选题目的
\objective{
  历史遗留系统通常存在有数据库仅有唯一存储的问题,导致一旦在数据库存储所在地发生不可抗力灾害会造成数据的部分或全部丢失,从而造成了经济损失。通过异地容灾备份数据库的方式,在灾害发生后可以尽快地恢复大部分的数据内容,减少经济损失。

  此外,历史遗留系统在备份的时候往往采用全量备份,此过程会造成极大的资源占用,降低了生产效率。而通过合理的全量备份及增量备份策略,可以极大地减少备份数据量,降低资源占用率与占用时间,从而提高生产效率。

  通过以上两种备份策略,能够有效地保护数据,提高系统的安全性。
}

% 思路
\methodology{
1. 实现异地容灾备份方案,一旦主数据存储无法使用(由于各种原因)立刻启用容灾数据存储提供服务;\\
2.  在主要数据库以及备份数据库之间通过多次快速的增量备份,异步保持数据的一致性。
}

% 研究方法/程序/步骤
\researchProcedure{
1. 使用阿里云部署阿里云主要系统服务与主数据存储(所选地区与容灾备份不相同),在本地部署备用系统服务与容灾备份数据存储,当主数据存储无法使用时,使系统服务指向备用数据存储; \\
2. 编写自动化脚本,定时地进行主数据存储到备份数据存储的增量备份与数据整合; \\
3. 针对案例系统修改配置,使的当主数据存储无法使用时自动切换使用备用数据存储。
}

% 相关支持条件
\supportment{
阿里云服务(包括用于部署系统服务的 ECS 实例若干以及数据存储实例) \\
容灾服务器 \\
相关数据库的备份技术 \\
案例系统的部署配置方案
}

% 进度安排
\schedule{
第一阶段(2016 年 11 月 27 日~2016 年 12 月 18 日): 对相关资料进行收集和翻阅,并对已搜集的资料加以整理,论证分析论文的可行性、实际性,将论文题目和大致范围确定下来,进行开题报告; \\
第二阶段(2016 年 12 月 19 日~2016 年 12 月 25 日):整合已有资料、构筑论文的大纲
第三阶段(2016 年 12 月 26 日~2017 年 3 月 6 日):根据查找的数据和相关资料,进行深入详实的论文编写工作,对论文编写过程中所发现的问题,研究其解决方案,推敲整合,并进行修改完善,准备论文中期检查 \\
2017 年 3 月 6 日 交中期检查报告; \\
第四阶段(2017 年 3 月 6 日~2017 年 4 月 5 日)完成论文的初稿部分,向指导老师寻求意见,优化论文的结构,润色语句,修改不当之处,补充不足之处。 \\
2017 年 4 月 5 日 提交初稿; \\
第四阶段(2017 年 4 月 5 日~2017 年 5 月 1 日):论文资料整合,最终定稿 \\
2017 年 5 月 1 日 定稿。
}

\proposalInstructions{
    测试指导老师意见
}

