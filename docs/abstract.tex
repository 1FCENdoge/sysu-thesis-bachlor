%%
% 摘要信息
% 本文档中前缀"c-"代表中文版字段, 前缀"e-"代表英文版字段
% 摘要内容应概括地反映出本论文的主要内容,主要说明本论文的研究目的、内容、方法、成果和结论。要突出本论文的创造性成果或新见解,不要与引言相 混淆。语言力求精练、准确,以 300—500 字为宜。
% 在摘要的下方另起一行,注明本文的关键词(3—5 个)。关键词是供检索用的主题词条,应采用能覆盖论文主要内容的通用技术词条(参照相应的技术术语 标准)。按词条的外延层次排列(外延大的排在前面)。摘要与关键词应在同一页。
% modifier: 黄俊杰(huangjj27, 349373001dc@gmail.com)
% update date: 2017-04-15
%%

\cabstract{
摘要内容应概括地反映出本论文的主要内容,主要说明本论文的研究目的、内容、方法、成果和结论。要突出本论文的创造性成果或新见解,不要与引言相混淆。语言力求精练、准确,以300—500字为宜。在摘要的下方另起一行,注明本文的关键词(3—5个)。关键词是供检索用的主题词条,应采用能覆盖论文主要内容的通用技术词条(参照相应的技术术语标准)。按词条的外延层次排列(外延大的排在前面)。摘要与关键词应在同一页。
}
% 中文关键词(每个关键词之间用“;”分开,最后一个关键词不打标点符号。)
\ckeywords{本科毕业论文;\LaTeX\ 模板;中山大学}

\eabstract{
% 英文摘要内容与中文摘要相同,以250—400个实词为宜。摘要下方另起一行注明英文关键词(Keywords3—5个)。
The content of the English abstract is the same as the Chinese abstract, 250-400 content words are appropriate. Start another line below the abstract to indicate English keywords (Keywords 3-5).
}
% 英文文关键词(每个关键词之间用;分开, 最后一个关键词不打标点符号。)
\ekeywords{undergraduate thesis; \LaTeX\ template; Sun Yat-Sen University}

